\documentclass[b5paper,fleqn]{ltjsarticle}
\usepackage{datenumber,amsthm,amsmath,latexsym,graphicx,hyperref}
\title{数学デーin大阪}
\newcounter{numcount}
\newcommand\tit[1]{\newpage\noindent{\bf\large 第\thenumcount 回} ---
#1\hfill\datedate\addtocounter{datenumber}{7}\setdatebynumber{\thedatenumber}\vskip1pt}
\newtheorem{theo}{定理}
\newtheorem{defi}[theo]{定義}
\newtheorem{prop}[theo]{命題}
\newtheorem{exam}[theo]{例}
\newtheorem*{proo}{証明}
\newtheorem{lemm}[theo]{補題}
\newcommand\image{\vskip18pt\includegraphics[width=.8\hsize]{resource/\thenumcount.jpg}\vskip5pt}
\newcommand\contents{{\bf 内容紹介:\vskip5pt}\stepcounter{numcount}}
\begin{document}
\maketitle
\begin{abstract}
本文章は数学デーの解説と、過去の活動を記録することを目的とする。
\end{abstract}\vskip30pt
まずは定義を与えよう。
\defi{数学デーとは数学を楽しむ同好の士が集う場、及びその日をいう。}
\begin{exam}$\phi$カフェ数学デー, みらいけん数学デー, 数学デーin大阪, 数学デーin札幌\end{exam}
\begin{prop}数学デーとは理系の人が参加するものである。\end{prop}
長らく議論された命題である。一見正しそうではあるが、東京や大阪で明らかな反例が見つかったことにより否定的に解決された。
\begin{theo}数学デーは楽しい\end{theo}
\begin{proo}この定理に関して、私は真に驚くべき証明を見つけたが、ここに書くには狭すぎる。\Box\end{proo}
数学デーin大阪では、開催毎にサブタイトルと数式を設定している。次頁からこれまでの活動を紹介する。
\setdatenumber{2019}{1}{12}\tit{積分ができるとは限らない}
\[n=p+q, \exists p,q:\text{prime number}\]
\image
\contents
ゴールドバッハの予想、整数論でオススメの本
\setdatenumber{2019}{1}{18}\tit{普遍性を求めて}
\[[n]_q:=\frac{1-q^n}{1-q}\]
\image
\contents
数学デー公式とSkype、バーゼル問題、立方体の万華鏡、計算尺の紹介、イデアル、素イデアル、絶対数学、測度論、量子コンピュータ、圏論でオススメの本、初めによむべき黒川先生の本は?1/3で割るってどういうこと?
\tit{全ての概念はKan拡張である}
\[\eta\left(-\frac{1}{\tau}\right):=\sqrt{\frac{\tau}{i}}\eta(\tau)\]
\image
\contents
直交多項式、\TeX をwebページで使うには?、mathjax、群論超入門、灘中の入試問題、\par
2/10+2/35+4/77+2/143の簡単な解き方\par
原価60円110円で売ると200個売れる。
一円下げると売れる数が10個増える法則があるとき利益が最大になるのは何円で売る時か?
こんな問題で甥っ子が答えられなくて問題文から数式が思いつかないみたい。
ここにいる人なら変数xにしてマイナスの二次関数になるからグラフの頂点を求めよって事なんだろと
予想つくだろうけど無理みたいなんです。
どう教えたらいいと思いますか?ってふると慣れしかないとか対応表で理解させて、
こんな面倒な方法じゃない方法もあるよーと教えるとかかな?みたいなアドバイスくれたって話。\par
100枚のコインがあって表90枚裏10枚になっているが被験者は裏表を識別できない。
その状態のまま2グループに分けてそれぞれの裏のコインの枚数を同一にするにはどうすればよいか?
\tit{マテーマタ「学ばれるべきもの」}
\[\zeta(s)=\sum^\infty_{n=1}\frac{1}{n^s}\]
\image
\contents
シャドウクローン、幾何学、台形の面積、\par
素数大富豪めちゃ楽しかった(メンバー全員4桁以上に挑み過ぎでしかも素数を引き当ててた)
\tit{君の頭は営業中かね?}
\[\lim_{x\to\infty}\frac{\operatorname{Li}x}{\pi(x)}=1\]
comming soon

\newpage
\thispagestyle{empty}
\vspace*{\stretch{1}}
\begin{flushright}
\begin{minipage}{0.5\hsize}
  \begin{tabular}{|ll}
  \multicolumn{2}{|l}{数学デーin大阪 運営}\\[10pt]
  名前: &\href{https://twitter.com/ThomsonKernel}{西村一輝/ThomsonKernel}\\
  所属: &\href{https://sites.google.com/view/osaka-dtc}{大阪分散技術コミュニティ}\\
  連絡先: & \href{mailto:thomsonkernel@gmail.com}{thomsonkernel@gmail.com}\\
  \\
  名前: &\href{https://twitter.com/myao_s_moking}{宮尾哲亮/myao}\\
  所属: &\href{https://kansai-lisp-useres.connpass.com}{関西Lispユーザ会}\\
  連絡先: & \href{mailto:tetu60u@yahoo.co.jp}{tetu60u@yahoo.co.jp}
  \end{tabular}
\end{minipage}
\end{flushright}

\end{document}
