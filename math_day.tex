\documentclass[b5paper,fleqn]{ltjsarticle}
\usepackage{datenumber,amsthm,amsmath,amssymb,latexsym,graphicx,hyperref}
\usepackage{luatex85}\usepackage[all]{xy}
\title{数学デーin大阪}
\newcounter{numcount}
\newcommand\tit[1]{\newpage\noindent{\bf\large 第\thenumcount 回} ---
#1\hfill\datedate\vskip1pt}
\newtheorem{theo}{定理}
\newtheorem{defi}[theo]{定義}
\newtheorem{prop}[theo]{命題}
\newtheorem{exam}[theo]{例}
\newtheorem*{proo}{証明}
\newtheorem{lemm}[theo]{補題}
\newcommand\image{\vskip18pt\includegraphics[width=.95\hsize]{resource/\thenumcount.jpg}\vskip5pt}
\newcommand\contents{{\bf 内容紹介:\vskip5pt}\stepcounter{numcount}}
\begin{document}
\maketitle
\begin{abstract}
本文章は数学デーの解説と、過去の活動を記録することを目的とする。
\end{abstract}\vskip30pt
まずは定義を与えよう。
\defi{数学好きまたは数学好きでない人がなんとなく集まる部室みたいなやつ}
\begin{exam}$\phi$カフェ数学デー, ソノリテ数学デー, 数学デーin N高, 数学デーin大阪
%みらいけん数学デー, 数学デーin埼玉, 数学デーin札幌, 
\end{exam}
\begin{prop}数学デーとは理系の人が参加するものである。\end{prop}
長らく議論された命題である。一見正しそうではあるが、東京や大阪で明らかな反例が見つかったことにより否定的に解決された。
\begin{theo}数学デーは楽しい\end{theo}
\begin{proo}この定理に関して、私は真に驚くべき証明を見つけたが、ここに書くには狭すぎる。\Box\end{proo}
数学デーin大阪では、開催毎にサブタイトルと数式を設定している。次頁からこれまでの活動を紹介する。

\setdatenumber{2019}{1}{12}
\tit{積分ができるとは限らない}
\[\forall n\geq4 \mbox{:even}, \exists p,q\mbox{:prime number},\mbox{such that}\quad n=p+q\]
\image
\contents
ゴールドバッハの予想、整数論でオススメの本,myaoさんのお父さんが書いたパズルの本

\setdatenumber{2019}{1}{18}
\tit{普遍性を求めて}
\[[n]_q:=\frac{1-q^n}{1-q}\]
\image
\contents
数学デー公式とSkype、バーゼル問題、立方体の万華鏡、計算尺の紹介、イデアル、素イデアル、絶対数学、測度論、量子コンピュータ、圏論でオススメの本、初めによむべき黒川先生の本は?1/3で割るってどういうこと?

\setdatenumber{2019}{1}{25}
\tit{全ての概念はKan拡張である}
\[\eta\left(-\frac{1}{\tau}\right):=\sqrt{\frac{\tau}{i}}\eta(\tau)\]
\image
\contents
直交多項式,\TeX をwebページで使うには?,mathjax,群論超入門,灘中の入試問題,2/10+2/35+4/77+2/143の簡単な解き方,
原価60円110円で売ると200個売れる。
一円下げると売れる数が10個増える法則があるとき利益が最大になるのは何円で売る時か?
こんな問題で甥っ子が答えられなくて問題文から数式が思いつかないみたい。
ここにいる人なら変数xにしてマイナスの二次関数になるからグラフの頂点を求めよって事なんだろと
予想つくだろうけど無理みたいなんです。
どう教えたらいいと思いますか?ってふると慣れしかないとか対応表で理解させて、
こんな面倒な方法じゃない方法もあるよーと教えるとかかな?みたいなアドバイスくれたって話,
100枚のコインがあって表90枚裏10枚になっているが被験者は裏表を識別できない。
その状態のまま2グループに分けてそれぞれの裏のコインの枚数を同一にするにはどうすればよいか?

\setdatenumber{2019}{2}{1}
\tit{マテーマタ「学ばれるべきもの」}
\[\zeta(s)=\sum^\infty_{n=1}\frac{1}{n^s}\]
\image
\contents
シャドウクローン、幾何学、台形の面積、\par
素数大富豪めちゃ楽しかった(メンバー全員4桁以上に挑み過ぎでしかも素数を引き当ててた)

\setdatenumber{2019}{2}{8}
\tit{君の頭は営業中かね?}
\[\lim_{x\to\infty}\frac{\operatorname{Li}x}{\pi(x)}=1\]
\image
\contents
簿記,D類似,線形代数,二項係数を拡張したい,一般項を求めたい

\setdatenumber{2019}{2}{15}
\tit{エウレカ「分かったぞ」}
\[\varepsilon_q\eta_p+\sigma_q\eta_p+\sigma_p\varepsilon_q\geq\frac{h}{4\pi}\]
\image
\contents
オセロ,ディープラーニング,xbox one,数学デーin埼玉とビデオ通話,スターリングの公式

\setdatenumber{2019}{2}{16}
\tit{何も仮定しません(Aucune hypothèse)}
\[V(I)=\{P\in\operatorname{Spec}(A)\ |\ I\subseteq P\}\]
\image
\contents
圏論, \href{https://twitter.com/setta243b}{Ee工房}さん持ち込みのパズルゲームが解けるかを
みんなで考えた、イプシロンデルタ事始め

\setdatenumber{2019}{2}{22}
\tit{万物は数である}
\[a^n+b^n=c^n\]
\image
\contents
コリドール(ボードゲーム)のベーシックストラテジー、ローラン展開、ブール代数(特にDNFなど)、
素数大富豪、フェルマー素数、CFT/CFT

\setdatenumber{2019}{3}{1}
\tit{空想は知識より重要である}
\[G_{\mu\nu}+\Lambda g_{\mu\nu}=\frac{8\pi G}{c^4}T_{\mu\nu}\]
\image
\contents
ガイスター(ボードゲーム)、ルベーグ積分入門、キネクトの台を設置した、東京とSkype通話、参加者が10人を超えたよ!!

\setdatenumber{2019}{3}{6}
\tit{数学者とは不正確な図を見ながら正確な推論のできる人間のことである\\}\vskip-15pt
\[S^3:=\{q\in \mathcal{H}\ |\ \|q\|=1\}\]
\image
\contents
条件付き確率、DIYおじさん、エンタングリオンのやり方を誰か教えて、弱い相互作用とは

\setdatenumber{2019}{3}{8}
\tit{ゼータに想いを馳せて}
\[\zeta(s)=\prod_{p:prime}\frac{1}{1-p^{-s}}\]
\image
\contents
やまだせんせい特別講義「ゼータ関数とリーマン予想」、数樂先生によるルベーグ積分、オセロマスターたみゅ、
東京組参戦(ひらうーさん)

\setdatenumber{2019}{3}{13}
\tit{3次曲面の上には27本の直線がある}
\[\{V\ |\ \operatorname{dim}(V\cap\mathbb{C}^{\omega(j)})\geq j\}\]
\image
\contents
エンタングリオンのやり方がようやく分かった、対称な直行多項式

\setdatenumber{2019}{3}{15}
\tit{ランダウは疲れることがどういうことかまったく知らなかった}
\[f(x)=\mathcal{O}(x^2)\]
\image
\contents
たけのこ赤軍による「ゼータ関数とリーマン予想(おまけ)」、ロマ数京都練習発表、onewanさん来阪

\setdatenumber{2019}{3}{20}
\tit{よって\(\heartsuit\)は値を持つ}
\[\vartheta(z;\tau):=\sum^\infty_{n=-\infty}\exp(\pi in^2\tau+2\pi inz)\]
\image
\contents
ウィルソンの定理の拡張、ソノリテ数学デーでも同じことをやっている、シグマ加法族、DYIおじさん再び

\setdatenumber{2019}{3}{22}
\tit{自然数は神がつくりたもうた。その他は人の業である。}
\[\sum_{k}\delta_{ik}\delta_{kj}=\delta_{ij}\]
\image
\contents
イロノワさんによる圏論講義、あよあんさんによる東京Skype講義、みねらる君の誕生日、参加者がはじめて20人となりました、
bukuさん来阪、
\setdatenumber{2019}{3}{27}
\tit{私は、数学上の大きな進歩に先鞭をつけた者が50歳を超えていた例を知らない。\\}
\[E=mc^2\]
\image
\contents
今日は今までで一番まったり、写真も撮るのを忘れた模様

\setdatenumber{2019}{3}{29}
\tit{グロタンディークは元気だが、あいかわらずだれにも会いたがらない\\}
\[e^{i\pi}+1=0\]
\image
\contents
\phi カフェお疲れ様でした、収束半径、ガンマ関数の一般化、角度を求める問題

\setdatenumber{2019}{3}{30}
\tit{自然科学における数学の不合理なまでの有効性について}
\[W(x,p):=\frac{1}{\pi\hbar}\int_{-\infty}^\infty \psi^*(x+y)\psi(x-y)e^{2ipy/\hbar}\,dy\]
\image
\contents
続・角度を求める問題、終了後は親睦会(たこ焼きパーティ)

\setdatenumber{2019}{4}{3}
\tit{We will rebuild the detector. There is no question.}
\[\pi^+\rightarrow \mu^++\nu_\mu\]
\image
\contents
宇佐美さんによるLie代数講義、しおんさん参戦、オセロ王陥落

\setdatenumber{2019}{4}{5}
\tit{そのようなBは腐るほどある}
\[A<B<2^A\]
\image
\contents
宇佐美さんによるLie代数講義2、数樂さんによるリーマン積分講義、オセロ王復冠

\setdatenumber{2019}{4}{10}
\tit{テーブルとイスとビールジョッキ}
\[y^2=x(x−a^n)(x+b^n)\]
\image
\contents
やまだせんせいによる例の問題解説、オセロ、きのこたけのこ戦争

\setdatenumber{2019}{4}{12}
\tit{玲瓏なる境地}
\[\oint_{\gamma}f(z)dz=0\]
\image
\contents
たけのこ赤軍によるドラームコホモロジー、補足おじさん、べき集合とは、全単射とは、高校数学からやり直す

\setdatenumber{2019}{4}{17}
\tit{私の勲章をさしあげますから、年金をまわしてくれませんかね}
\[f(z_1,z_2,\cdots,z_n)\]
\image
\contents
ハーツホーンがわからない、集合トランプ、おばけキャッチ

\setdatenumber{2019}{4}{19}
\tit{数学とは異なるものを同じものとみなす技術である。}
\[A\simeq B\]
\image
\contents
たけのこ赤軍による代数幾何入門、素数大富豪、ノートが綺麗、高校生から分かる複素解析

\setdatenumber{2019}{4}{24}
\tit{圏の定義のAha!的でないありがたさ}
\[\xymatrix{A\ar@{->}[r]^{\textbf{1}_A}\ar@/^6mm/[rr]^{\textbf{1}_A}
\ar@/_6mm/[rr]_{\textbf{1}_A'}&A\ar@{->}[r]^{\textbf{1}_A'}&A}\]
\image
\contents
下から積み上げる四目並べ、曲線Cの体K上のJacobi多様体

\setdatenumber{2019}{4}{26}
\tit{スペクトル系列三羽烏に何ができる}
\[\operatorname{Ker}(f^n)/\operatorname{Im}(f^{n-1})\]
\image
\contents
代数学の基本定理、カタラン数について、キャベツ太郎

\setdatenumber{2019}{5}{8}
\tit{天を翻し地を覆さんと、我が心は高ぶる}
\[x\equiv a\pmod{m}\]
\image
\contents
ゆるハーツホーン、おばけキャッチ、素粒子論事始め

\setdatenumber{2019}{5}{10}
\tit{強い相互作用の理論における漸近的自由性に対して}
\[\beta(\alpha_s) = \mu^2 \frac{\partial \alpha_s(\mu^2)}{\partial \mu^2}\]
\image
\contents
Oddieさん参戦、ライツアウトパズル、数独

\setdatenumber{2019}{5}{15}
\tit{ケシ粒を日に日に倍にすると120日で何粒になるか}
\[y=2^x\]
\image
\contents
爆弾脱出ゲーム、茎

\setdatenumber{2019}{5}{17}
\tit{素数大富豪で遊ぼう}
\[2, 3, 5, 7, 11, 13, 17, 19, 23, 29, 31, 37, 41, 43, 47, 53, 57\]
\image
\contents
素数大富豪

\setdatenumber{2019}{5}{22}
\tit{tan1°は無理数である}
\[\varphi(30)=\pi(30)=8\]
\image
\contents
ひたすら超幾何級数の比をテイラー展開してました、川村先生arXiv掲載おめでとう

\setdatenumber{2019}{5}{24}
\tit{もっと素数大富豪で遊ぼう}
\[\sigma(2^5\times31)=2^6\times31\]
\image
\contents
いきなり熱い展開

\setdatenumber{2019}{5}{29}
\tit{オセロ王に挑戦しよう}
\[2^32+1=641\times6700417\]
\image
\contents
たみゅさん9連戦、タギロン、主張に矛盾がある、多重ガンマ関数入門

\setdatenumber{2019}{5}{31}
\tit{ほとんどすべての整数は4つの立方数の和であらわされる}
\[8866128975287528^3-8778405442862239^3-2736111468807040^3=33\]
\image
\contents
超難問模試を考える

\setdatenumber{2019}{6}{5}
\tit{(正の)フィボナッチ数に平方数は2つしかない}
\[1,1,2,3,5,8,13,21,34,55,89,144\]
\image
\contents
二平方和定理、レルヒの公式、物性の研究、タイムラプスをやってみる

\setdatenumber{2019}{6}{7}
\tit{素数全体の集合は任意の長さの等差数列を含む}
\[n(2,3)=35\]
\image
\contents
今日も元気にやっています、海老江数理科学勉強会発足

\setdatenumber{2019}{6}{12}
\tit{任意のヘキソミノは平面を埋め尽くす}
\image
\contents
金曜日の準備をしていました

\setdatenumber{2019}{6}{14}
\tit{非正則素数は無限に多く存在する}
\image
\contents
解析数論講義

\setdatenumber{2019}{6}{19}
\tit{それらの研究は "q-病" 呼ばわりされるほどに伝染的であった}
\image
\contents
量子解析とテータ特殊値、おかえり数樂さん

\setdatenumber{2019}{6}{21}
\tit{特徴のない最小の数という特徴}
\image
\contents
凸関数について、解析の問題を解く、Goodfellow読書会

\setdatenumber{2019}{6}{26}
\tit{定規とコンパスによる一般角の3等分は不可能である}
\image
\contents
イロノワさんによる計算機科学と数学の交わり,
入口のボードですがウェットティッシュで消せることが判明しました。
ただし消した後乾いたティッシュで拭かないと書いた後が滲んでしまいます。

\setdatenumber{2019}{6}{28}
\tit{類数1の虚2次体は9個しかない}
\image
\contents
$2^n+1$の形の完全数が存在しないことの証明、Tの射影さん降臨、一様収束と各点収束を理解した

\setdatenumber{2019}{7}{3}
\tit{種数2の非特異代数曲線の有理点は有限個である}
\contents
coming soon
\setdatenumber{2019}{7}{5}
\tit{ほとんどの有理数は3つの単位分数の逆数の和であらわされる}
\contents
coming soon


%\tit{尋常ならざる剛性}
%\tit{みんな数学してる。数学してないのはお前だけ。}
%tit{4次曲面の上には、直線があるとは限らない。}
%tit{物理屋になりたかったんだよ}
%tit{ご冗談でしょうファインマンさん}
%tit{それは我が愛する青春の夢です。}
%tit{ユークリッドのアルゴリズム}
%tit{檄文到らば 即日打ち立たれるべし}
%tit{Love and Math - The Heart of Hidden Reality}
%tit{ζの化身に姿をかえて}
%tit{ゆっくりと急げ(Festina lente)}
%tit{数学は愛のチャージ(荷)を担うことができる}
%tit{足し算と掛け算を分離する}
%tit{数学は紳士のゲームだ}
%tit{数学は紳士のゲームだ}
%\tit{ドーナツはなんでもできるんだなぁ}


\newpage
\thispagestyle{empty}
\vspace*{\fill}
\begin{flushright}
\begin{minipage}{0.5\hsize}
  \begin{tabular}{|ll}
  \multicolumn{2}{|l}{数学デーin大阪 運営}\\[10pt]
  名前: &\href{https://twitter.com/ThomsonKernel}{西村一輝/ThomsonKernel}\\
  所属: &\href{https://sites.google.com/view/osaka-dtc}{大阪分散技術コミュニティ}\\
  連絡先: & \href{mailto:thomsonkernel@gmail.com}{thomsonkernel@gmail.com}\\
  \\
  名前: &\href{https://twitter.com/691_7758337633}{たけのこ赤軍(川村先生)}\\
  所属: &高校生数学者\\
  \\
  名前: &\href{https://twitter.com/tyamada1093}{Tomohiro Yamada}\\
  所属: &\href{https://kansai-lisp-useres.connpass.com}{数学者、大学非常勤講師}\\
  \\
  名前: &\href{https://twitter.com/coc_mathfun}{数樂}\\
  所属: &大学生\\
  名前: &\href{https://twitter.com/yoriyuki}{yoriyuki}\\
  所属: &研究者\\
  \end{tabular}
  \begin{tabular}{l}
\end{tabular}\vskip15pt
Special thanks\\
   \href{https://twitter.com/notori48}{Kuma@酒飲み}
   , \href{https://twitter.com/ONEWAN}{onewan}
   , \href{https://twitter.com/Paya_payashi}{パヤシ}\vskip10pt
   \href{https://twitter.com/sugaku_day}{数学デー公式}\\
   \href{https://twitter.com/kiguro_masanao}{キグロ}
   , \href{https://twitter.com/euchaeta}{Euchaeta}
   , \href{https://twitter.com/motcho_tw}{鯵坂もっちょ}
\end{minipage}
\end{flushright}

\end{document}
