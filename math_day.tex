\documentclass[b5paper,fleqn]{ltjsarticle}
\usepackage{datenumber}
\title{数学デーin大阪}
\author{大阪分散技術コミュニティ}
\setdatenumber{2019}{1}{18}\newcounter{numcount}
\newcommand\tit[1]{\stepcounter{numcount}\vskip10pt\noindent{\bf\large 第\thenumcount 回} ---
#1\hfill\datedate\vskip5pt \addtocounter{datenumber}{7}\setdatebynumber{\thedatenumber}}
\begin{document}
\maketitle
\begin{abstract}
本文章は数学デーの定義を与え、過去の活動を記録することを目的とする。
\end{abstract}\vskip30pt
\tit{普遍性を求めて}
\[[n]_q:=\frac{1-q^n}{1-q}\]
\tit{全ての概念はKan拡張である}
\[\eta\left(-\frac{1}{\tau}\right):=\sqrt{\frac{\tau}{i}}\eta(\tau)\]
\tit{マテーマタ「学ばれるべきもの」}
\[\zeta(s)=\sum^\infty_{n=1}\frac{1}{n^s}\]
comming soon

\end{document}
